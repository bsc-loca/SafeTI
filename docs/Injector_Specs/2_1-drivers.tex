\subsection{Designing software for traffic generation and control}
\label{software-drivers}

The available function calls are organized into three categories; \fullref{software-drivers-program}, \fullref{software-drivers-setup} and \fullref{software-drivers-memfunc}.

\subsubsection{Injection program example}
\label{software-drivers-program}

The default driver files include a injection program example as the \textbf{inj\_program()} function, which can be reviewed on the Appendix 
\fullref{appendix-programexample}. 
Using this function, it can be understood the main points to know about when designing an injection program. 
This program is used for the program setup that generates transactions of a particular type, while inserting \textbf{DELAY} between transaction batches if desired. 
First, let's understand the variables the \textbf{inj\_program()} function admits as input.

\begin{itemize}
  \item \textbf{DESC\_N\_BATCH}: The number of descriptors the injection program will contain. This variable must be at least 2 if inserting DELAY descriptors is desired.
  \item \textbf{DESC\_TYPE}: The type of traffic to be generated.
  \item \textbf{INJ\_QUEUE}: Asserts the \textbf{QUEUE\_MODE\_EN} parameter during the configuration of the injector.
  \item \textbf{SIZE\_RD\_WR}: Number of bytes that are accessed during the execution of a descriptor iteration of \textbf{READ}, \textbf{WRITE}, \textbf{READ\_FIX} or \textbf{WRITE\_FIX} type.
  \item \textbf{DESC\_ATTACK\_ADDR}: Pointer of the starting address where the traffic being generated accesses.
  \item \textbf{SIZE\_DELAY}: Number of clock cycles to be on standby inserted between \textbf{DESC\_TYPE} descriptor types.
\end{itemize}
\vspace{15px}

At the start of the function body, there's a number of initialization variables that use specific structs of the driver. 
These are used to set up the appropriated configuration of the injector and manage the different descriptors to upload on the injector as an injection program. 
The \autoref{table:structs} compiles the different parameters to be set on each of these structs variables.

\begin{table}[h]
  \begin{tabular}{@{}p{0.15\linewidth}p{0.15\linewidth}x{0.2\linewidth}p{0.5\linewidth}@{}}
    \toprule
    \multicolumn{1}{c}{Struct name} & \multicolumn{1}{c}{Parameter name} & \multicolumn{1}{c}{Admissible value range} & \multicolumn{1}{c}{Description of the parameter}                                                              \\
    \cmidrule{1-4}
    \multirow{6}{*}{desc\_ctrl}       & last           & {[}0-1{]}            & Last descriptor flag on injection program.                                                                                                            \\
                                      & type           & {[}0-6{]}            & Descriptor type encoding. Use the descriptor name type listed on \autoref{table:descriptors} while adding \textbf{INJ\_OP\_} as a prefix.             \\
                                      & irq\_compl\_en & {[}0-1{]}            & Send an APB interruption on descriptor execution completion.                                                                                          \\
                                      & count          & {[}0-63{]}           & Set a number of descriptor iterations equal to \textbf{COUNT}+1.                                                                                      \\
                                      & size           & {[}1-524288{]}       & Set the number of bytes to access or the number of clock cycles to wait on standby depending on the descriptor type on a single execution iteration.  \\
    &&& \\
    \multirow{2}{*}{desc\_delay}      & ctrl           & Struct               & Nested struct parameter of type \textbf{desc\_ctrl}.                                                                                                  \\
    &&& \\
    \multirow{2}{*}{desc\_rd\_wr}     & ctrl           & Struct               & Nested struct parameter of type \textbf{desc\_ctrl}.                                                                                                  \\
                                      & act\_addr      & {[}0x0-0xFFFFFFFF{]} & Variable that sets the ATTACK\_ADDR on \textbf{READ}, \textbf{WRITE}, \textbf{READ\_FIX} and \textbf{WRITE\_FIX} descriptor types.                    \\
    \bottomrule
  \end{tabular}
\caption{Struct types used on the injector configuration and programming.}
\label{table:structs}
\end{table}

The \textbf{inj\_config} struct has been omitted from the \autoref{table:structs} due to containing the same parameters previously explained on 
\autoref{table:configuration}.

After setting the desired structs, namely an \textbf{inj\_config} for the injector configuration, \textbf{desc\_rd\_wr} for the traffic generation and 
\textbf{desc\_delay} in case the insertion of \textbf{DELAY} descriptors is desired between transaction descriptors, the function proceeds by loading the 
descriptors in execution order to the injector and by last the injector configuration using setup calls.
These are explained on \fullref{software-drivers-setup}.


\subsubsection{Injection setup functions}
\label{software-drivers-setup}

The setup functions are used to parse the struct variables to actual 32-bit words following the machine format the injector understands, to then be written 
on the injector using the memory access functions presented on \autoref{software-drivers-memfunc}, or also to perform control functions like a status check.
The following list compiles all available setup functions.

\begin{itemize}
  \item \textbf{inj\_setup()}: It accepts a \textbf{inj\_config} struct variable pointer as input, which reorganizes and writes on the injector.
  \item \textbf{inj\_reset()}: Doesn't have any input, setting the configuration to zeros, except the \textbf{RESET\_SW} parameter that is set to 1, resetting the injector.
  \item \textbf{inj\_check\_run()}: Doesn't have any input. This function returns 1 if the actual injector configuration has \textbf{ENABLE} parameter asserted, meaning it is running.
  \item \textbf{setup\_descriptor\_control()}: It accepts a \textbf{desc\_ctrl} struct variable pointer as input, which reorganizes and writes on the injector as word of a descriptor. This call should never be used on an injector program directly, since it is more organized to use it on a specific descriptor type setup calls.
  \item \textbf{setup\_descriptor\_delay()}: It accepts a \textbf{desc\_delay} struct variable pointer as input, which reorganizes and writes on the injector as word of a descriptor of \textbf{DELAY} type.
  \item \textbf{setup\_descriptor\_rd\_wr()}: It accepts a \textbf{desc\_rd\_wr} struct variable pointer as input, which reorganizes and writes on the injector as words of a descriptor of \textbf{READ}, \textbf{WRITE}, \textbf{READ\_FIX} or \textbf{WRITE\_FIX} types.
\end{itemize}
\vspace{15px}


\subsubsection{Injection memory access functions}
\label{software-drivers-memfunc}

The memory access functions are functions used to execute read and writes on the APB memory space allocated for the injector.
A difference between the memory access and the setup functions is that these only work with 32-bit data words without knowledge on the actual application of the 
access.
It is recommended, for best organization, to use these following memory access functions through setup functions and not directly on an injection program.

\begin{itemize}
  \item \textbf{inj\_read\_reg()}:
  \item \textbf{inj\_write\_reg()}:
\end{itemize}
\vspace{15px}

The memory access functions have a limitation imposed where the access is limited to the APB memory space used to allocate the injector, providing low-level 
protection against erroneous address access while using the injector drivers. 
This is achieved by using the variables \textbf{APB\_MEM\_SPACE} and \textbf{INJ\_BASE\_ADDR} as a limiter and pointer to the base address of the injector 
allocation on the APB memory space.
Both variables are located on the driver file \textbf{injector.h} and must be updated accordingly to the implementation.

