\newpage
\section{Module description}
\label{module desc}

The SafeTI IP act as a programmable traffic injector on the network which the component is implemented on.
The injector's programming and configuration is carried out through writes and reads on the APB Slave registers of the component.
The APB interface it's implemented as an internal core component of the injector, contrary to the interface of the network where traffic is generated, 
which is interchangeable during implementation for easy adaptation.
This is better presented looking at \fullref{fig:injector_big}, where exemplifies the compulsory APB bus connectivity and an interchangeable interface 
where the traffic is injected.

At the moment, the supported platforms only include SELENE, and the supported injection networks are both AHB and AXI4. However, the former AHB requires an 
interface component from the platform in order to operate correctly.

This specification manual is meant to aid on the use, implementation and expansion of the SafeTI.
Thus, since different tasks requires different knowledge, this manual is written in three sections; \fullref{software}, 
\fullref{} and \fullref{}. 2. Importing the SafeTI IP onto a new platform, 3. Modifying core components of the injector.

However, it is important first to acknowledge the traffic injector's main features categorized as \fullref{module desc-commands} and \fullref{module desc-config}.


\subsection{Injector commands}
\label{module desc-commands}

An injection program is a list of descriptors, also called injection vectors, that are executed in sequential succession by the traffic injector. 
The traffic injector features the following descriptor types specified on \autoref{table:descriptors}.

\begin{table}[h]
  \begin{tabular}{@{}p{0.15\linewidth}x{0.2\linewidth}p{0.65\linewidth}@{}}
    \toprule
    \multicolumn{1}{x{0.15\linewidth}}{Descriptor type} & \multicolumn{1}{x{0.2\linewidth}}{Specific type parameters} & \multicolumn{1}{x{0.65\linewidth}}{Description of the execution}                                                                    \\
    \cmidrule{1-3}
    DELAY       & COUNT, SIZE               & Equivalent to a no operation instruction. For \textbf{SIZE} times \textbf{COUNT}+1 clock cycles, the injector remains on standby.                                                                             \\
    READ        & COUNT, SIZE, ATTACK\_ADDR & Traffic generation of sequential read transactions for accessing a minimum of \textbf{SIZE} bytes, starting from the address \textbf{ATTACK\_ADDR}. This operation is executed \textbf{COUNT}+1 iterations.   \\
    WRITE       & COUNT, SIZE, ATTACK\_ADDR & Traffic generation of sequential write transactions for accessing a minimum of \textbf{SIZE} bytes, starting from the address \textbf{ATTACK\_ADDR}. This operation is executed \textbf{COUNT}+1 iterations.  \\
    READ\_FIX   & COUNT, SIZE, ATTACK\_ADDR & Traffic generation of read transactions for accessing a minimum of \textbf{SIZE} bytes, continuously on the address \textbf{ATTACK\_ADDR}. This operation is executed \textbf{COUNT}+1 iterations.            \\
    WRITE\_FIX  & COUNT, SIZE, ATTACK\_ADDR & Traffic generation of write transactions for accessing a minimum of \textbf{SIZE} bytes, continuously on the address \textbf{ATTACK\_ADDR}. This operation is executed \textbf{COUNT}+1 iterations.           \\
    \bottomrule
  \end{tabular}
\caption{Available descriptor types, its specific parameters and execution description.}
\label{table:descriptors}
\end{table}


In addition, there's a number of common parameters available which functionality is the same for every descriptor. 
These are \textbf{LAST} and \textbf{IRQ\_COMPL\_EN} that when asserted (set to 1), the former acts as last descriptor on the injection program, while the latter 
sets an interruption pulse through the APB bus when the descriptor has been fully executed. 
By default, the injector will halt on program completion set by executing a descriptor with the variable \textbf{LAST} asserted.

The injector program must be compiled and set up through the APB bus to the injector previous to enabling it. 
Though it is possible to add descriptors to the injector program during its execution, it's not recommended due to possible early termination of the program 
or incorrect execution.


\subsection{Injector configuration}
\label{module desc-config}

The traffic injector also features a general configuration that is applied during the execution of the injector program conformed by the aforementioned 
descriptors. These are compiled on \autoref{table:configuration}. 
Any injector configuration can be overwritten at any moment, even during the execution of an injection program.

\begin{table}[h]
  \begin{tabular}{@{}p{0.2\linewidth}p{0.8\linewidth}@{}}
    \toprule
    \multicolumn{1}{c}{Configuration parameter} & \multicolumn{1}{c}{Description of the parameter when asserted}                                                                                                       \\
    \cmidrule{1-2}
      ENABLE                & Start or resume execution of the loaded injector program. Setting it to 0 allows to halt the execution of the injection program, though active transactions will be allowed to be finished. \\
      RESET\_SW             & Reset all configuration and injector program execution. Ongoing transactions will continue to end properly.                                                                                 \\
      QUEUE\_MODE\_EN       & Loop execution to first descriptor on program completion meaning, the injector will not stop executing the program after completing a descriptor with the LAST variable asserted.           \\
      IRQ\_PROG\_COMPL\_EN  & Send an APB interruption on program completion.                                                                                                                                             \\
      IRQ\_ERR\_CORE\_EN    & Send an APB interruption on error from the internal components of the injector.                                                                                                             \\
      IRQ\_ERR\_NET\_EN     & Send an APB interruption on error from the network where the traffic is being generated.                                                                                                    \\
      FREEZE\_IRQ\_EN       & Halt program execution at any interruption. This includes interruption due to description parameter \textbf{IRQ\_COMPL\_EN}.                                                                 \\
    \bottomrule
  \end{tabular}
\caption{Injector configuration and its functionality.}
\label{table:configuration}
\end{table}
