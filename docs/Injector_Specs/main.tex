% License:
% CC BY-NC-SA 3.0 (http://creativecommons.org/licenses/by-nc-sa/3.0/)
%
%%%%%%%%%%%%%%%%%%%%%%%%%%%%%%%%%%%%%%%%%

%----------------------------------------------------------------------------------------
%	PACKAGES AND OTHER DOCUMENT CONFIGURATIONS
%----------------------------------------------------------------------------------------

\documentclass[paper=a4, fontsize=11pt]{scrartcl} % A4 paper and 11pt font size

\usepackage[T1]{fontenc} % Use 8-bit encoding that has 256 glyphs
\usepackage{fourier} % Use the Adobe Utopia font for the document - comment this line to return to the LaTeX default
\usepackage[english]{babel} % English language/hyphenation
\usepackage{amsmath,amsfonts,amsthm} % Math packages

\usepackage{caption}
\usepackage{subcaption}
\usepackage{graphicx}

\usepackage{float}

\usepackage{blindtext} %for enumarations

\usepackage{hyperref}
\newcommand*{\fullref}[1]{\hyperref[{#1}]{\ref*{#1} \nameref*{#1}}} % Command for referencing section number and name.

%talbe layout to the right
%\usepackage[labelfont=bf]{caption}
%\captionsetup[table]{labelsep=space,justification=raggedright,singlelinecheck=off}
%\captionsetup[figure]{labelsep=quad}

\usepackage{sectsty} % Allows customizing section commands
\allsectionsfont{\centering \normalfont\scshape} % Make all sections centered, the default font and small caps

\usepackage{fancyhdr} % Custom headers and footers
\pagestyle{fancyplain} % Makes all pages in the document conform to the custom headers and footers
\fancyhead{} % No page header - if you want one, create it in the same way as the footers below
\fancyfoot[L]{} % Empty left footer
\fancyfoot[C]{} % Empty center footer
\fancyfoot[R]{\thepage} % Page numbering for right footer
\renewcommand{\headrulewidth}{0pt} % Remove header underlines
\renewcommand{\footrulewidth}{0pt} % Remove footer underlines
\setlength{\headheight}{13.6pt} % Customize the height of the header

\numberwithin{equation}{section} % Number equations within sections (i.e. 1.1, 1.2, 2.1, 2.2 instead of 1, 2, 3, 4)
\numberwithin{figure}{section} % Number figures within sections (i.e. 1.1, 1.2, 2.1, 2.2 instead of 1, 2, 3, 4)
\numberwithin{table}{section} % Number tables within sections (i.e. 1.1, 1.2, 2.1, 2.2 instead of 1, 2, 3, 4)

%\setlength\parindent{0pt} % Removes all indentation from paragraphs - comment this line for an assignment with lots of text


\setlength\parskip{4pt}


%----------------------------------------------------------------------------------------
% TABLE PACKAGES
%----------------------------------------------------------------------------------------

\usepackage{colortbl}
\usepackage{multirow} % Used in tables
\usepackage{booktabs}
\usepackage{array}
\renewcommand{\arraystretch}{1.2}
\newcolumntype{x}[1]{>{\centering\let\newline\\\arraybackslash\hspace{0pt}}p{#1}}


%----------------------------------------------------------------------------------------
% DIGITAL REGISTER TABLES
%----------------------------------------------------------------------------------------

% Rights to Matthew Lowell (lovells@gmail.com), "Register diagrams with field descriptions", 2020/03/22 
% Register tables import package
\usepackage{register} %Register library

% Homebrew command for creating text label bellow the register figure.
\newcommand{\regfieldtext}[3]{%
  % Compute basic width of a single bit
  \settowidth{\regFieldLen}{\regLabelSize \regResetName}%
  \setlength{\regFieldLen}{\regWidth - \regFieldLen}%
  \setlength{\regFieldLen}{\regFieldLen / \regBitWidth}%
  % Figure out height and depth of reset field in current font
  % Is there a more ‘‘official’’ method to do this?
  \settodepth{\regResetDepth}{\regResetSize ()jgpq}%
  \settoheight{\regResetHeight}{\regResetSize ()ABCjkl}%
  %\addtolength{\regResetHeight}{\regResetDepth}%
  % Compute how far to drop the reset fields down. The value at
  % the end is effectively the separation between the bit position
  % box and the reset value box.
  %\setlength{\regResetDrop}{\regResetHeight + 2\fboxsep - 2\fboxrule + 3pt}%
  % New lengths to support colorbox use, when fboxsep gets set to 0
  %\setlength{\regRsvdDrop}{\regResetDrop + \fboxsep}%
  %\setlength{\regRsvdHeight}{\regResetHeight + 2\fboxsep}%
  \setlength{\regFieldLen}{#2\regFieldLen + \fboxrule}%
  % Figure out bit positions
  %\setcounter{lowerbit}{#3}%
  %\setcounter{upperbit}{#3 + #2 - 1}%
  % Typeset reset value in a framebox
  \makebox[0pt][l]{\raisebox{-0.5\regResetDrop}{\makebox[\regFieldLen][c]%
  % Place an invisible rule to control the box
  % surrounding the reset field
  {\regResetSize \rule[-1\regResetDepth]{0pt}{\regResetHeight}\normalsize#1}}}%
  % Typeset bit positions in a framebox
  \ifthenelse{#2 > 1}%
    {\makebox[\regFieldLen][c]%
    {\regBitSize \rule[-1\regResetDepth]{0pt}{\regResetHeight}%
    \regBitFamily \hfill}}%
  {\makebox[\regFieldLen][c]%
  {\regBitSize \rule[-1\regResetDepth]{0pt}{\regResetHeight}%
  \regBitFamily}}%
  \hspace{-1\fboxrule}%
}

% Homebrew command for creating text label on the left.
\newcommand{\reglabelc}[1]{%
  %\hspace{6px}
  \settowidth{\regFieldLen}{\regLabelSize \regResetName}%
  $\,$\raisebox{-2.5\fboxsep}{\makebox[\regFieldLen][l]%
  {\regLabelSize\regBitFamily\textbf{#1}}}%
}


%----------------------------------------------------------------------------------------
% CODE PACKAGES AND COLOR OPTIONS
%----------------------------------------------------------------------------------------

\usepackage{listings}
\usepackage{xcolor}

\definecolor{codegreen}{rgb}{0,0.6,0}
\definecolor{codegray}{rgb}{0.5,0.5,0.5}
\definecolor{codepurple}{rgb}{0.58,0,0.82}
\definecolor{backcolour}{rgb}{0.95,0.95,0.92}

\lstdefinestyle{mystyle}{
    backgroundcolor=\color{backcolour},   
    commentstyle=\color{codegreen},
    keywordstyle=\color{magenta},
    numberstyle=\tiny\color{codegray},
    stringstyle=\color{codepurple},
    basicstyle=\ttfamily\footnotesize,
    breakatwhitespace=false,         
    breaklines=true,                 
    captionpos=b,                    
    keepspaces=true,                 
    numbers=left,                    
    numbersep=5pt,                  
    showspaces=false,                
    showstringspaces=false,
    showtabs=false,                  
    tabsize=2
}

\lstset{style=mystyle}


%----------------------------------------------------------------------------------------
%	TITLE SECTION
%----------------------------------------------------------------------------------------

\newcommand{\horrule}[1]{\rule{\linewidth}{#1}} % Create horizontal rule command with 1 argument of height

\title{	
\normalfont \normalsize 
\horrule{0.5pt} \\[0.4cm] % Thin top horizontal rule
\Huge  SafeTI Manual \\ % The assignment title
\vspace{10px}
\LARGE Traffic Injector Specifications \\ % Subtitle
\horrule{0.5pt} \\[0.4cm] % Thick bottom horizontal rule
}

\author{Francisco Javier Fuentes Diaz} % Your name
\date{\today} % Today's date or a custom date

\graphicspath {{img}} % Figure folder

\begin{document}
\nocite{*}
\maketitle % Print the title

\newpage
\tableofcontents

\newpage
\section{Module description}
\label{module desc}

The SafeTI IP act as a programmable traffic injector on the network which the component is implemented on.
The injector's programming and configuration is carried out through writes and reads on the APB Slave registers of the component.
The APB interface it's implemented as an internal core component of the injector, contrary to the interface of the network where traffic is generated, 
which is interchangeable during implementation for plug-n-play adaptation to the network if there's an available compatible interface.
This is better presented looking at \autoref{fig:safeti_interface}, where exemplifies the compulsory APB bus connectivity and an interchangeable interface 
where the traffic is injected.

At the moment, the supported platforms only include SELENE, and the supported injection networks are both AHB and AXI4. However, the former AHB requires an 
interface component from the platform in order to operate correctly.

\begin{figure}[h]
  \includegraphics[height=120px]{{safeti_interface.png}}
  \centering
  \caption{Block diagram of the SafeTI core connected to a compatible Network interface using the interface bus (IB), while showing the connections with the 
  APB network for control and programming and the Network where the traffic is generated.}
  \label{fig:safeti_interface}
\end{figure}

This specification manual is meant to aid on the use, implementation and expansion of the SafeTI components and software.
Thus, since different tasks requires different knowledge, this manual is written in three sections; \fullref{software}, 
\fullref{} and \fullref{}. 2. Importing the SafeTI IP onto a new platform, 3. Modifying core components of the injector.

However, it is important first to acknowledge the traffic injector's main features categorized as \fullref{module desc-commands} and \fullref{module desc-config}.


\subsection{Injector commands}
\label{module desc-commands}

An injection program is a list of descriptors, also called injection vectors, that are executed in sequential succession by the traffic injector. 
The traffic injector features the following descriptor types specified on \autoref{table:descriptors}.

\begin{table}[h]
  \begin{tabular}{@{}p{0.15\linewidth}x{0.2\linewidth}p{0.65\linewidth}@{}}
    \toprule
    \multicolumn{1}{x{0.15\linewidth}}{Descriptor type} & \multicolumn{1}{x{0.2\linewidth}}{Specific type parameters} & \multicolumn{1}{x{0.65\linewidth}}{Description of the execution}                                                                    \\
    \cmidrule{1-3}
    DELAY       & COUNT, SIZE               & Equivalent to a no operation instruction. For \textbf{SIZE} times \textbf{COUNT}+1 clock cycles, the injector remains on standby.                                                                             \\
    READ        & COUNT, SIZE, ATTACK\_ADDR & Traffic generation of sequential read transactions for accessing a minimum of \textbf{SIZE} bytes, starting from the address \textbf{ATTACK\_ADDR}. This operation is executed \textbf{COUNT}+1 iterations.   \\
    WRITE       & COUNT, SIZE, ATTACK\_ADDR & Traffic generation of sequential write transactions for accessing a minimum of \textbf{SIZE} bytes, starting from the address \textbf{ATTACK\_ADDR}. This operation is executed \textbf{COUNT}+1 iterations.  \\
    READ\_FIX   & COUNT, SIZE, ATTACK\_ADDR & Traffic generation of read transactions for accessing a minimum of \textbf{SIZE} bytes, continuously on the address \textbf{ATTACK\_ADDR}. This operation is executed \textbf{COUNT}+1 iterations.            \\
    WRITE\_FIX  & COUNT, SIZE, ATTACK\_ADDR & Traffic generation of write transactions for accessing a minimum of \textbf{SIZE} bytes, continuously on the address \textbf{ATTACK\_ADDR}. This operation is executed \textbf{COUNT}+1 iterations.           \\
    \bottomrule
  \end{tabular}
\caption{Available descriptor types, its specific parameters and execution description.}
\label{table:descriptors}
\end{table}


In addition, there's a number of common parameters available which functionality is the same for every descriptor. 
These are the parameters \textbf{LAST} and \textbf{IRQ\_COMPL\_EN} that when enabled, the former acts as last descriptor on 
the injection program while the latter sets an interruption pulse through the APB bus when the descriptor has been fully executed. 
By default, the injector will halt on program completion by executing a descriptor with the parameter \textbf{LAST} enabled.

The injector program must be compiled and set up through the APB bus to the injector previous to enabling it. 
Though it is possible to add descriptors to the injector program during its execution, it is not recommended due to possible early termination of the program 
or incorrect execution.


\subsection{Injector configuration}
\label{module desc-config}

The traffic injector also features a general configuration that is applied during the execution of the injector program conformed by the aforementioned 
descriptors. These are compiled on \autoref{table:configuration}. 
Any injector configuration can be overwritten at any moment, even during the execution of an injection program.

\begin{table}[h]
  \begin{tabular}{@{}p{0.2\linewidth}p{0.8\linewidth}@{}}
    \toprule
    \multicolumn{1}{c}{Configuration parameter} & \multicolumn{1}{c}{Description of the parameter when asserted}                                                                                                                                                                      \\
    \cmidrule{1-2}
      ENABLE                & Start or resume execution of the loaded injector program. Disabling this parameter halts the execution of the injection program, though active transactions will be allowed to be finish.                                                               \\
      RESET\_SW             & Reset all SafeTI components and injector program execution. Ongoing transactions will be allowed to be finish, but the configuration set by the call that makes the reset is sustained.                                                                 \\
      QUEUE\_MODE\_EN       & Loop execution to first descriptor on program completion. Meaning, when enabled, the injector will return to first descriptor of the program when completing a descriptor with the parameter \textbf{LAST} enabled.                                     \\
      IRQ\_PROG\_COMPL\_EN  & When enabled, send an APB interruption on program completion.                                                                                                                                                                                           \\
      IRQ\_ERR\_CORE\_EN    & When enabled, send an APB interruption on error from the internal components of the injector.                                                                                                                                                           \\
      IRQ\_ERR\_NET\_EN     & When enabled, send an APB interruption on error from the network where the traffic is being generated.                                                                                                                                                  \\
      FREEZE\_IRQ\_EN       & Halt program execution at any interruption. This includes interruption due to enabling description parameter \textbf{IRQ\_COMPL\_EN}, injector configuration \textbf{IRQ\_PROG\_COMPL\_EN}, \textbf{IRQ\_ERR\_CORE\_EN} or \textbf{IRQ\_ERR\_NET\_EN}.  \\
    \bottomrule
  \end{tabular}
\caption{Injector configuration parameters function.}
\label{table:configuration}
\end{table}


\input{2-section.tex}

\newpage
\section{Appendixes}
\subsection{Example injection program}
\label{appendix-programexample}

\begin{lstlisting}[language=Verilog, caption=Example injection program function from the bare-metal driver file \textbf{injector.c}.]
//*** Template Program Functions ***/

// Number of descriptors        / Type of transactions / Enable Queue mode                      /
// Transfer size per descriptor / R/W address          / Delay weigth after transfer descriptor /
void inj_program( unsigned int DESC_N_BATCH, unsigned int DESC_TYPE, unsigned int INJ_QUEUE, unsigned int SIZE_RD_WR, unsigned int DESC_ATTACK_ADDR, unsigned int SIZE_DELAY ) {
  unsigned int i  = 0;  // Loop variable
  inj_config inj_cfg;   // Injector configuration
  desc_ctrl  ctrl_word; // Descriptor control word
  desc_delay delay;     // Descriptor for DELAY type
  desc_rd_wr rd_wr;     // Descriptor for READ and WRTIE types

  // Check if the injector is running and reset it if it is.
  if(inj_check_run() != 0) {
    inj_reset();
    printf("WARNING: The injector was running before programming it. Thus, it has been reset.\n");
  }

  // Injector configuration
  inj_cfg.enable            = 1;
  inj_cfg.reset             = 0;
  if(INJ_QUEUE == 1)
    inj_cfg.queue_mode_en   = 1;
  else
    inj_cfg.queue_mode_en   = 0;
  inj_cfg.irq_prog_compl_en = 0;
  inj_cfg.irq_err_core_en   = 1;
  inj_cfg.irq_err_net_en    = 1;
  inj_cfg.freeze_irq_en     = 0;

  // Common descriptor setup
  ctrl_word.irq_compl_en    = 0;

  // Descriptor generation and setup
  for(i = 0; i < DESC_N_BATCH; i++) {

    ctrl_word.irq_compl_en = 0;     // Enable interrupt on descriptor completion
    ctrl_word.count   = 0;          // Number of repetitions of the descriptor

    if(i == DESC_N_BATCH - 1)       // Last descriptor bit
      ctrl_word.last  = 1;
    else
      ctrl_word.last  = 0;

    // Insert DELAY descriptor between two READ/WRITE descriptors if SIZE_DELAY has a value
    if(SIZE_DELAY != 0 & i%2 == 1) {
      ctrl_word.type  = INJ_OP_DELAY;
      ctrl_word.size  = SIZE_DELAY;
      delay.ctrl      = ctrl_word;
      setup_descriptor_delay(&delay);
    } else {
      ctrl_word.type  = DESC_TYPE;
      ctrl_word.size  = SIZE_RD_WR;
      rd_wr.ctrl      = ctrl_word;
      rd_wr.act_addr  = DESC_ATTACK_ADDR;
      setup_descriptor_rd_wr(&rd_wr);
    }

  }

  // Launch the injector with the configuration
  inj_run(&inj_cfg);
}

\end{lstlisting}



\end{document}
