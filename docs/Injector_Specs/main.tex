% License:
% CC BY-NC-SA 3.0 (http://creativecommons.org/licenses/by-nc-sa/3.0/)
%
%%%%%%%%%%%%%%%%%%%%%%%%%%%%%%%%%%%%%%%%%

%----------------------------------------------------------------------------------------
%	PACKAGES AND OTHER DOCUMENT CONFIGURATIONS
%----------------------------------------------------------------------------------------

\documentclass[paper=a4, fontsize=11pt]{scrartcl} % A4 paper and 11pt font size

\usepackage[T1]{fontenc} % Use 8-bit encoding that has 256 glyphs
\usepackage{fourier} % Use the Adobe Utopia font for the document - comment this line to return to the LaTeX default
\usepackage[english]{babel} % English language/hyphenation
\usepackage{amsmath,amsfonts,amsthm} % Math packages

\usepackage{caption}
\usepackage{subcaption}
\usepackage{graphicx}

\usepackage{float}

\usepackage{blindtext} %for enumarations

\usepackage{hyperref}
\newcommand*{\fullref}[1]{\hyperref[{#1}]{\ref*{#1} \nameref*{#1}}} % Command for referencing section number and name.

%talbe layout to the right
%\usepackage[labelfont=bf]{caption}
%\captionsetup[table]{labelsep=space,justification=raggedright,singlelinecheck=off}
%\captionsetup[figure]{labelsep=quad}

\usepackage{sectsty} % Allows customizing section commands
\allsectionsfont{\centering \normalfont\scshape} % Make all sections centered, the default font and small caps

\usepackage{fancyhdr} % Custom headers and footers
\pagestyle{fancyplain} % Makes all pages in the document conform to the custom headers and footers
\fancyhead{} % No page header - if you want one, create it in the same way as the footers below
\fancyfoot[L]{} % Empty left footer
\fancyfoot[C]{} % Empty center footer
\fancyfoot[R]{\thepage} % Page numbering for right footer
\renewcommand{\headrulewidth}{0pt} % Remove header underlines
\renewcommand{\footrulewidth}{0pt} % Remove footer underlines
\setlength{\headheight}{13.6pt} % Customize the height of the header

\numberwithin{equation}{section} % Number equations within sections (i.e. 1.1, 1.2, 2.1, 2.2 instead of 1, 2, 3, 4)
\numberwithin{figure}{section} % Number figures within sections (i.e. 1.1, 1.2, 2.1, 2.2 instead of 1, 2, 3, 4)
\numberwithin{table}{section} % Number tables within sections (i.e. 1.1, 1.2, 2.1, 2.2 instead of 1, 2, 3, 4)

%\setlength\parindent{0pt} % Removes all indentation from paragraphs - comment this line for an assignment with lots of text


\setlength\parskip{4pt}


%----------------------------------------------------------------------------------------
% TABLE PACKAGES AND COLOR OPTIONS
%----------------------------------------------------------------------------------------

\usepackage{multirow} % Used in tables
\usepackage{booktabs}
\usepackage{array}
\renewcommand{\arraystretch}{1.2}
\newcolumntype{x}[1]{>{\centering\let\newline\\\arraybackslash\hspace{0pt}}p{#1}}


%----------------------------------------------------------------------------------------
% CODE PACKAGES AND COLOR OPTIONS
%----------------------------------------------------------------------------------------

\usepackage{listings}
\usepackage{xcolor}

\definecolor{codegreen}{rgb}{0,0.6,0}
\definecolor{codegray}{rgb}{0.5,0.5,0.5}
\definecolor{codepurple}{rgb}{0.58,0,0.82}
\definecolor{backcolour}{rgb}{0.95,0.95,0.92}

\lstdefinestyle{mystyle}{
    backgroundcolor=\color{backcolour},   
    commentstyle=\color{codegreen},
    keywordstyle=\color{magenta},
    numberstyle=\tiny\color{codegray},
    stringstyle=\color{codepurple},
    basicstyle=\ttfamily\footnotesize,
    breakatwhitespace=false,         
    breaklines=true,                 
    captionpos=b,                    
    keepspaces=true,                 
    numbers=left,                    
    numbersep=5pt,                  
    showspaces=false,                
    showstringspaces=false,
    showtabs=false,                  
    tabsize=2
}

\lstset{style=mystyle}


%----------------------------------------------------------------------------------------
%	TITLE SECTION
%----------------------------------------------------------------------------------------

\newcommand{\horrule}[1]{\rule{\linewidth}{#1}} % Create horizontal rule command with 1 argument of height

\title{	
\normalfont \normalsize 
\horrule{0.5pt} \\[0.4cm] % Thin top horizontal rule
\Huge  SafeTI Manual \\ % The assignment title
\vspace{10px}
\LARGE Traffic Injector Specifications \\ % Subtitle
\horrule{0.5pt} \\[0.4cm] % Thick bottom horizontal rule
}

\author{Francisco Javier Fuentes Diaz} % Your name
\date{\today} % Today's date or a custom date

\begin{document}
\nocite{*}
\maketitle % Print the title

\newpage
\tableofcontents

\subsection{Module description}
\label{module_desc}
This unit acts as an AHB/AXI Master IP connected to the main bus on the SELENE Platform. It acts as a core with limited capabilities, only generating transactions to the bus by reading and writing to the AHB Slave RAM memory and controlled via APB registers.\\
The injector works along with the multi-core setup instantiated on the platform and other peripherals and monitoring units.\\
The module's specifications described in this section include non-implemented features, and it might be revisited for future improvements.\\
In order to generate traffic to the bus, the module performs a set of data transactions based on descriptors set at startup into a predefined memory address range.\\
The Traffic Injector is based on generic Direct Memory Access functionality fundamentals and extends its features to meet the injector objectives.\\
The internal components and respective configuration and functionalities are described in next sections.




\newpage
\section{Designing software for traffic generation and control}
\label{software}

The SafeTI IP includes bare-metal drivers located on \verb|bsc\_inj/sw| project folder, providing the foundation for the injection program the user is designing. 
It is required that both files \textbf{injector.c} and \textbf{injector.h} are included on the main program that uses the injector calls, usually as library files.

This section supposes that the SafeTI is already implemented on the desired platform, meaning it has a fixed \textbf{INJ\_BASE\_ADDR}, which is the base address 
where the injector is allocated as an APB Slave component. This parameter must be updated accordingly on the \textbf{injector.h} file, so the calls are executed 
on the memory space the injector is allocated. In the case the \textbf{INJ\_BASE\_ADDR} is unknown, refer to section \fullref{} for further information.

The following section \fullref{software-drivers} presents the available calls by default on the drivers used to design an injection program.
However, they're limited in scope, thus, section \fullref{software-baremetal} provides the required information to expand the drivers at the user desire.

\subsection{Designing software for traffic generation and control}
\label{software-drivers}

The available function calls are organized into three categories; \fullref{software-drivers-program}, \fullref{software-drivers-setup} and \fullref{software-drivers-memfunc}.

\subsubsection{Injection program example}
\label{software-drivers-program}

The default driver files include a injection program example as the \textbf{inj\_program()} function, which can be reviewed on the Appendix 
\fullref{appendix-programexample}. 
Using this function, it can be understood the main points to know about when designing an injection program. 
This program is used for the program setup that generates transactions of a particular type, while inserting \textbf{DELAY} between transaction batches if desired. 
First, let's understand the variables the \textbf{inj\_program()} function admits as input.

\begin{itemize}
  \item \textbf{DESC\_N\_BATCH}: The number of descriptors the injection program will contain. This variable must be at least 2 if inserting DELAY descriptors is desired.
  \item \textbf{DESC\_TYPE}: The type of traffic to be generated.
  \item \textbf{INJ\_QUEUE}: Asserts the \textbf{QUEUE\_MODE\_EN} parameter during the configuration of the injector.
  \item \textbf{SIZE\_RD\_WR}: Number of bytes that are accessed during the execution of a descriptor iteration of \textbf{READ}, \textbf{WRITE}, \textbf{READ\_FIX} or \textbf{WRITE\_FIX} type.
  \item \textbf{DESC\_ATTACK\_ADDR}: Pointer of the starting address where the traffic being generated accesses.
  \item \textbf{SIZE\_DELAY}: Number of clock cycles to be on standby inserted between \textbf{DESC\_TYPE} descriptor types.
\end{itemize}
\vspace{15px}

At the start of the function body, there's a number of initialization variables that use specific structs of the driver. 
These are used to set up the appropriated configuration of the injector and manage the different descriptors to upload on the injector as an injection program. 
The \autoref{table:structs} compiles the different parameters to be set on each of these structs variables.

\begin{table}[h]
  \begin{tabular}{@{}p{0.15\linewidth}p{0.15\linewidth}x{0.2\linewidth}p{0.5\linewidth}@{}}
    \toprule
    \multicolumn{1}{c}{Struct name} & \multicolumn{1}{c}{Parameter name} & \multicolumn{1}{c}{Admissible value range} & \multicolumn{1}{c}{Description of the parameter}                                                              \\
    \cmidrule{1-4}
    \multirow{6}{*}{desc\_ctrl}       & last           & {[}0-1{]}            & Last descriptor flag on injection program.                                                                                                            \\
                                      & type           & {[}0-6{]}            & Descriptor type encoding. Use the descriptor name type listed on \autoref{table:descriptors} while adding \textbf{INJ\_OP\_} as a prefix.             \\
                                      & irq\_compl\_en & {[}0-1{]}            & Send an APB interruption on descriptor execution completion.                                                                                          \\
                                      & count          & {[}0-63{]}           & Set a number of descriptor iterations equal to \textbf{COUNT}+1.                                                                                      \\
                                      & size           & {[}1-524288{]}       & Set the number of bytes to access or the number of clock cycles to wait on standby depending on the descriptor type on a single execution iteration.  \\
    &&& \\
    \multirow{2}{*}{desc\_delay}      & ctrl           & Struct               & Nested struct parameter of type \textbf{desc\_ctrl}.                                                                                                  \\
    &&& \\
    \multirow{2}{*}{desc\_rd\_wr}     & ctrl           & Struct               & Nested struct parameter of type \textbf{desc\_ctrl}.                                                                                                  \\
                                      & act\_addr      & {[}0x0-0xFFFFFFFF{]} & Variable that sets the ATTACK\_ADDR on \textbf{READ}, \textbf{WRITE}, \textbf{READ\_FIX} and \textbf{WRITE\_FIX} descriptor types.                    \\
    \bottomrule
  \end{tabular}
\caption{Struct types used on the injector configuration and programming.}
\label{table:structs}
\end{table}

The \textbf{inj\_config} struct has been omitted from the \autoref{table:structs} due to containing the same parameters previously explained on 
\autoref{table:configuration}.

After setting the desired structs, namely an \textbf{inj\_config} for the injector configuration, \textbf{desc\_rd\_wr} for the traffic generation and 
\textbf{desc\_delay} in case the insertion of \textbf{DELAY} descriptors is desired between transaction descriptors, the function proceeds by loading the 
descriptors in execution order to the injector and by last the injector configuration using setup calls.
These are explained on \fullref{software-drivers-setup}.


\subsubsection{Injection setup functions}
\label{software-drivers-setup}

The setup functions are used to parse the struct variables to actual 32-bit words following the machine format the injector understands, to then be written 
on the injector using the memory access functions presented on \autoref{software-drivers-memfunc}, or also to perform control functions like a status check.
The following list compiles all available setup functions.

\begin{itemize}
  \item \textbf{inj\_setup()}: It accepts a \textbf{inj\_config} struct variable pointer as input, which reorganizes and writes on the injector.
  \item \textbf{inj\_reset()}: Doesn't have any input, setting the configuration to zeros, except the \textbf{RESET\_SW} parameter that is set to 1, resetting the injector.
  \item \textbf{inj\_check\_run()}: Doesn't have any input. This function returns 1 if the actual injector configuration has \textbf{ENABLE} parameter asserted, meaning it is running.
  \item \textbf{setup\_descriptor\_control()}: It accepts a \textbf{desc\_ctrl} struct variable pointer as input, which reorganizes and writes on the injector as word of a descriptor. This call should never be used on an injector program directly, since it is more organized to use it on a specific descriptor type setup calls.
  \item \textbf{setup\_descriptor\_delay()}: It accepts a \textbf{desc\_delay} struct variable pointer as input, which reorganizes and writes on the injector as word of a descriptor of \textbf{DELAY} type.
  \item \textbf{setup\_descriptor\_rd\_wr()}: It accepts a \textbf{desc\_rd\_wr} struct variable pointer as input, which reorganizes and writes on the injector as words of a descriptor of \textbf{READ}, \textbf{WRITE}, \textbf{READ\_FIX} or \textbf{WRITE\_FIX} types.
\end{itemize}
\vspace{15px}


\subsubsection{Injection memory access functions}
\label{software-drivers-memfunc}

The memory access functions are functions used to execute read and writes on the APB memory space allocated for the injector.
A difference between the memory access and the setup functions is that these only work with 32-bit data words without knowledge on the actual application of the 
access.
It is recommended, for best organization, to use these following memory access functions through setup functions and not directly on an injection program.

\begin{itemize}
  \item \textbf{inj\_read\_reg()}:
  \item \textbf{inj\_write\_reg()}:
\end{itemize}
\vspace{15px}

The memory access functions have a limitation imposed where the access is limited to the APB memory space used to allocate the injector, providing low-level 
protection against erroneous address access while using the injector drivers. 
This is achieved by using the variables \textbf{APB\_MEM\_SPACE} and \textbf{INJ\_BASE\_ADDR} as a limiter and pointer to the base address of the injector 
allocation on the APB memory space.
Both variables are located on the driver file \textbf{injector.h} and must be updated accordingly to the implementation.



\newpage
\subsection{Bare-metal program and control}
\label{software-baremetal}






\newpage
\section{Appendixes}
\subsection{Example injection program}
\label{appendix-programexample}

\begin{lstlisting}[language=Verilog, caption=Example injection program function from the bare-metal driver file \textbf{injector.c}.]
//*** Template Program Functions ***/

// Number of descriptors        / Type of transactions / Enable Queue mode                      /
// Transfer size per descriptor / R/W address          / Delay weigth after transfer descriptor /
void inj_program( unsigned int DESC_N_BATCH, unsigned int DESC_TYPE, unsigned int INJ_QUEUE, unsigned int SIZE_RD_WR, unsigned int DESC_ATTACK_ADDR, unsigned int SIZE_DELAY ) {
  unsigned int i  = 0;  // Loop variable
  inj_config inj_cfg;   // Injector configuration
  desc_ctrl  ctrl_word; // Descriptor control word
  desc_delay delay;     // Descriptor for DELAY type
  desc_rd_wr rd_wr;     // Descriptor for READ and WRTIE types

  // Check if the injector is running and reset it if it is.
  if(inj_check_run() != 0) {
    inj_reset();
    printf("WARNING: The injector was running before programming it. Thus, it has been reset.\n");
  }

  // Injector configuration
  inj_cfg.enable            = 1;
  inj_cfg.reset             = 0;
  if(INJ_QUEUE == 1)
    inj_cfg.queue_mode_en   = 1;
  else
    inj_cfg.queue_mode_en   = 0;
  inj_cfg.irq_prog_compl_en = 0;
  inj_cfg.irq_err_core_en   = 1;
  inj_cfg.irq_err_net_en    = 1;
  inj_cfg.freeze_irq_en     = 0;

  // Common descriptor setup
  ctrl_word.irq_compl_en    = 0;

  // Descriptor generation and setup
  for(i = 0; i < DESC_N_BATCH; i++) {

    ctrl_word.irq_compl_en = 0;     // Enable interrupt on descriptor completion
    ctrl_word.count   = 0;          // Number of repetitions of the descriptor

    if(i == DESC_N_BATCH - 1)       // Last descriptor bit
      ctrl_word.last  = 1;
    else
      ctrl_word.last  = 0;

    // Insert DELAY descriptor between two READ/WRITE descriptors if SIZE_DELAY has a value
    if(SIZE_DELAY != 0 & i%2 == 1) {
      ctrl_word.type  = INJ_OP_DELAY;
      ctrl_word.size  = SIZE_DELAY;
      delay.ctrl      = ctrl_word;
      setup_descriptor_delay(&delay);
    } else {
      ctrl_word.type  = DESC_TYPE;
      ctrl_word.size  = SIZE_RD_WR;
      rd_wr.ctrl      = ctrl_word;
      rd_wr.act_addr  = DESC_ATTACK_ADDR;
      setup_descriptor_rd_wr(&rd_wr);
    }

  }

  // Launch the injector with the configuration
  inj_run(&inj_cfg);
}

\end{lstlisting}



\end{document}
