% License:
% CC BY-NC-SA 3.0 (http://creativecommons.org/licenses/by-nc-sa/3.0/)
%
%%%%%%%%%%%%%%%%%%%%%%%%%%%%%%%%%%%%%%%%%

%----------------------------------------------------------------------------------------
%	PACKAGES AND OTHER DOCUMENT CONFIGURATIONS
%----------------------------------------------------------------------------------------

\documentclass[paper=a4, fontsize=11pt]{scrartcl} % A4 paper and 11pt font size

\usepackage[T1]{fontenc} % Use 8-bit encoding that has 256 glyphs
\usepackage{fourier} % Use the Adobe Utopia font for the document - comment this line to return to the LaTeX default
\usepackage[english]{babel} % English language/hyphenation
\usepackage{amsmath,amsfonts,amsthm} % Math packages

\usepackage{caption}
\usepackage{subcaption}
\usepackage{graphicx}

\usepackage{float}

\usepackage{blindtext} %for enumarations

\usepackage{hyperref}
\newcommand*{\fullref}[1]{\hyperref[{#1}]{\ref*{#1} \nameref*{#1}}} % Command for referencing section number and name.

%talbe layout to the right
%\usepackage[labelfont=bf]{caption}
%\captionsetup[table]{labelsep=space,justification=raggedright,singlelinecheck=off}
%\captionsetup[figure]{labelsep=quad}

\usepackage{sectsty} % Allows customizing section commands
\allsectionsfont{\centering \normalfont\scshape} % Make all sections centered, the default font and small caps

\usepackage{fancyhdr} % Custom headers and footers
\pagestyle{fancyplain} % Makes all pages in the document conform to the custom headers and footers
\fancyhead{} % No page header - if you want one, create it in the same way as the footers below
\fancyfoot[L]{} % Empty left footer
\fancyfoot[C]{} % Empty center footer
\fancyfoot[R]{\thepage} % Page numbering for right footer
\renewcommand{\headrulewidth}{0pt} % Remove header underlines
\renewcommand{\footrulewidth}{0pt} % Remove footer underlines
\setlength{\headheight}{13.6pt} % Customize the height of the header

\numberwithin{equation}{section} % Number equations within sections (i.e. 1.1, 1.2, 2.1, 2.2 instead of 1, 2, 3, 4)
\numberwithin{figure}{section} % Number figures within sections (i.e. 1.1, 1.2, 2.1, 2.2 instead of 1, 2, 3, 4)
\numberwithin{table}{section} % Number tables within sections (i.e. 1.1, 1.2, 2.1, 2.2 instead of 1, 2, 3, 4)

%\setlength\parindent{0pt} % Removes all indentation from paragraphs - comment this line for an assignment with lots of text


\setlength\parskip{4pt}


%----------------------------------------------------------------------------------------
% TABLE PACKAGES
%----------------------------------------------------------------------------------------

\usepackage{colortbl}
\usepackage{multirow} % Used in tables
\usepackage{booktabs}
\usepackage{array}
\renewcommand{\arraystretch}{1.2}
\newcolumntype{x}[1]{>{\centering\let\newline\\\arraybackslash\hspace{0pt}}p{#1}}


%----------------------------------------------------------------------------------------
% DIGITAL REGISTER TABLES
%----------------------------------------------------------------------------------------

% Rights to Matthew Lowell (lovells@gmail.com), "Register diagrams with field descriptions", 2020/03/22 
% Register tables import package
\usepackage{register} %Register library

% Homebrew command for creating text label bellow the register figure.
\newcommand{\regfieldtext}[3]{%
  % Compute basic width of a single bit
  \settowidth{\regFieldLen}{\regLabelSize \regResetName}%
  \setlength{\regFieldLen}{\regWidth - \regFieldLen}%
  \setlength{\regFieldLen}{\regFieldLen / \regBitWidth}%
  % Figure out height and depth of reset field in current font
  % Is there a more ‘‘official’’ method to do this?
  \settodepth{\regResetDepth}{\regResetSize ()jgpq}%
  \settoheight{\regResetHeight}{\regResetSize ()ABCjkl}%
  %\addtolength{\regResetHeight}{\regResetDepth}%
  % Compute how far to drop the reset fields down. The value at
  % the end is effectively the separation between the bit position
  % box and the reset value box.
  %\setlength{\regResetDrop}{\regResetHeight + 2\fboxsep - 2\fboxrule + 3pt}%
  % New lengths to support colorbox use, when fboxsep gets set to 0
  %\setlength{\regRsvdDrop}{\regResetDrop + \fboxsep}%
  %\setlength{\regRsvdHeight}{\regResetHeight + 2\fboxsep}%
  \setlength{\regFieldLen}{#2\regFieldLen + \fboxrule}%
  % Figure out bit positions
  %\setcounter{lowerbit}{#3}%
  %\setcounter{upperbit}{#3 + #2 - 1}%
  % Typeset reset value in a framebox
  \makebox[0pt][l]{\raisebox{-0.5\regResetDrop}{\makebox[\regFieldLen][c]%
  % Place an invisible rule to control the box
  % surrounding the reset field
  {\regResetSize \rule[-1\regResetDepth]{0pt}{\regResetHeight}\normalsize#1}}}%
  % Typeset bit positions in a framebox
  \ifthenelse{#2 > 1}%
    {\makebox[\regFieldLen][c]%
    {\regBitSize \rule[-1\regResetDepth]{0pt}{\regResetHeight}%
    \regBitFamily \hfill}}%
  {\makebox[\regFieldLen][c]%
  {\regBitSize \rule[-1\regResetDepth]{0pt}{\regResetHeight}%
  \regBitFamily}}%
  \hspace{-1\fboxrule}%
}

% Homebrew command for creating text label on the left.
\newcommand{\reglabelc}[1]{%
  %\hspace{6px}
  \settowidth{\regFieldLen}{\regLabelSize \regResetName}%
  $\,$\raisebox{-2.5\fboxsep}{\makebox[\regFieldLen][l]%
  {\regLabelSize\regBitFamily\textbf{#1}}}%
}


%----------------------------------------------------------------------------------------
% CODE PACKAGES AND COLOR OPTIONS
%----------------------------------------------------------------------------------------

\usepackage{listings}
\usepackage{xcolor}

\definecolor{codegreen}{rgb}{0,0.6,0}
\definecolor{codegray}{rgb}{0.5,0.5,0.5}
\definecolor{codepurple}{rgb}{0.58,0,0.82}
\definecolor{backcolour}{rgb}{0.95,0.95,0.92}

\lstdefinestyle{mystyle}{
    backgroundcolor=\color{backcolour},   
    commentstyle=\color{codegreen},
    keywordstyle=\color{magenta},
    numberstyle=\tiny\color{codegray},
    stringstyle=\color{codepurple},
    basicstyle=\ttfamily\footnotesize,
    breakatwhitespace=false,         
    breaklines=true,                 
    captionpos=b,                    
    keepspaces=true,                 
    numbers=left,                    
    numbersep=5pt,                  
    showspaces=false,                
    showstringspaces=false,
    showtabs=false,                  
    tabsize=2
}

\lstset{style=mystyle}


%----------------------------------------------------------------------------------------
%	TITLE SECTION
%----------------------------------------------------------------------------------------

\newcommand{\horrule}[1]{\rule{\linewidth}{#1}} % Create horizontal rule command with 1 argument of height

\title{	
\normalfont \normalsize 
\horrule{0.5pt} \\[0.4cm] % Thin top horizontal rule
\Huge  SafeTI Manual \\ % The assignment title
\vspace{10px}
\LARGE Traffic Injector Specifications \\ % Subtitle
\horrule{0.5pt} \\[0.4cm] % Thick bottom horizontal rule
}

\author{Francisco Javier Fuentes Diaz} % Your name
\date{\today} % Today's date or a custom date

\graphicspath {{img}} % Figure folder

\begin{document}
\nocite{*}
\maketitle % Print the title

\newpage
\tableofcontents

\newpage
\section{Module description}
\label{module desc}

The SafeTI IP act as a programmable traffic injector on the network which the component is implemented on.
The injector's programming and configuration is carried out through writes and reads on the APB Slave registers of the component.
The APB interface it's implemented as an internal core component of the injector, contrary to the interface of the network where traffic is generated, 
which is interchangeable during implementation for plug-n-play adaptation to the network if there's an available compatible interface.
This is better presented looking at \autoref{fig:safeti_interface}, where exemplifies the compulsory APB bus connectivity and an interchangeable interface 
where the traffic is injected.

At the moment, the supported platforms only include SELENE, and the supported injection networks are both AHB and AXI4. However, the former AHB requires an 
interface component from the platform in order to operate correctly.

\begin{figure}[h]
  \includegraphics[height=120px]{{safeti_interface.png}}
  \centering
  \caption{Block diagram of the SafeTI core connected to a compatible Network interface using the interface bus (IB), while showing the connections with the 
  APB network for control and programming and the Network where the traffic is generated.}
  \label{fig:safeti_interface}
\end{figure}

This specification manual is meant to aid on the use, implementation and expansion of the SafeTI components and software.
Thus, since different tasks requires different knowledge, this manual is written in three sections; \fullref{software}, 
\fullref{} and \fullref{}. 2. Importing the SafeTI IP onto a new platform, 3. Modifying core components of the injector.

However, it is important first to acknowledge the traffic injector's main features categorized as \fullref{module desc-commands} and \fullref{module desc-config}.


\subsection{Injector commands}
\label{module desc-commands}

An injection program is a list of descriptors, also called injection vectors, that are executed in sequential succession by the traffic injector. 
The traffic injector features the following descriptor types specified on \autoref{table:descriptors}.

\begin{table}[h]
  \begin{tabular}{@{}p{0.15\linewidth}x{0.2\linewidth}p{0.65\linewidth}@{}}
    \toprule
    \multicolumn{1}{x{0.15\linewidth}}{Descriptor type} & \multicolumn{1}{x{0.2\linewidth}}{Specific type parameters} & \multicolumn{1}{x{0.65\linewidth}}{Description of the execution}                                                                    \\
    \cmidrule{1-3}
    DELAY       & COUNT, SIZE               & Equivalent to a no operation instruction. For \textbf{SIZE} times \textbf{COUNT}+1 clock cycles, the injector remains on standby.                                                                             \\
    READ        & COUNT, SIZE, ATTACK\_ADDR & Traffic generation of sequential read transactions for accessing a minimum of \textbf{SIZE} bytes, starting from the address \textbf{ATTACK\_ADDR}. This operation is executed \textbf{COUNT}+1 iterations.   \\
    WRITE       & COUNT, SIZE, ATTACK\_ADDR & Traffic generation of sequential write transactions for accessing a minimum of \textbf{SIZE} bytes, starting from the address \textbf{ATTACK\_ADDR}. This operation is executed \textbf{COUNT}+1 iterations.  \\
    READ\_FIX   & COUNT, SIZE, ATTACK\_ADDR & Traffic generation of read transactions for accessing a minimum of \textbf{SIZE} bytes, continuously on the address \textbf{ATTACK\_ADDR}. This operation is executed \textbf{COUNT}+1 iterations.            \\
    WRITE\_FIX  & COUNT, SIZE, ATTACK\_ADDR & Traffic generation of write transactions for accessing a minimum of \textbf{SIZE} bytes, continuously on the address \textbf{ATTACK\_ADDR}. This operation is executed \textbf{COUNT}+1 iterations.           \\
    \bottomrule
  \end{tabular}
\caption{Available descriptor types, its specific parameters and execution description.}
\label{table:descriptors}
\end{table}


In addition, there's a number of common parameters available which functionality is the same for every descriptor. 
These are the parameters \textbf{LAST} and \textbf{IRQ\_COMPL\_EN} that when enabled, the former acts as last descriptor on 
the injection program while the latter sets an interruption pulse through the APB bus when the descriptor has been fully executed. 
By default, the injector will halt on program completion by executing a descriptor with the parameter \textbf{LAST} enabled.

The injector program must be compiled and set up through the APB bus to the injector previous to enabling it. 
Though it is possible to add descriptors to the injector program during its execution, it is not recommended due to possible early termination of the program 
or incorrect execution.


\subsection{Injector configuration}
\label{module desc-config}

The traffic injector also features a general configuration that is applied during the execution of the injector program conformed by the aforementioned 
descriptors. These are compiled on \autoref{table:configuration}. 
Any injector configuration can be overwritten at any moment, even during the execution of an injection program.

\begin{table}[h]
  \begin{tabular}{@{}p{0.2\linewidth}p{0.8\linewidth}@{}}
    \toprule
    \multicolumn{1}{c}{Configuration parameter} & \multicolumn{1}{c}{Description of the parameter when asserted}                                                                                                                                                                      \\
    \cmidrule{1-2}
      ENABLE                & Start or resume execution of the loaded injector program. Disabling this parameter halts the execution of the injection program, though active transactions will be allowed to be finish.                                                               \\
      RESET\_SW             & Reset all SafeTI components and injector program execution. Ongoing transactions will be allowed to be finish, but the configuration set by the call that makes the reset is sustained.                                                                 \\
      QUEUE\_MODE\_EN       & Loop execution to first descriptor on program completion. Meaning, when enabled, the injector will return to first descriptor of the program when completing a descriptor with the parameter \textbf{LAST} enabled.                                     \\
      IRQ\_PROG\_COMPL\_EN  & When enabled, send an APB interruption on program completion.                                                                                                                                                                                           \\
      IRQ\_ERR\_CORE\_EN    & When enabled, send an APB interruption on error from the internal components of the injector.                                                                                                                                                           \\
      IRQ\_ERR\_NET\_EN     & When enabled, send an APB interruption on error from the network where the traffic is being generated.                                                                                                                                                  \\
      FREEZE\_IRQ\_EN       & Halt program execution at any interruption. This includes interruption due to enabling description parameter \textbf{IRQ\_COMPL\_EN}, injector configuration \textbf{IRQ\_PROG\_COMPL\_EN}, \textbf{IRQ\_ERR\_CORE\_EN} or \textbf{IRQ\_ERR\_NET\_EN}.  \\
    \bottomrule
  \end{tabular}
\caption{Injector configuration parameters function.}
\label{table:configuration}
\end{table}


\newpage
\section{Designing software for traffic generation and control}
\label{software}

The SafeTI IP includes bare-metal drivers, located on \textbf{bsc\_safeti/sw} project folder, providing the foundation for the injection program the user is designing. 
It is required that both files \textbf{injector.c} and \textbf{injector.h} are included on the main program that uses the injector calls, usually as library files.

This section supposes that the SafeTI is already implemented on the desired platform, meaning it has a fixed \textbf{INJ\_BASE\_ADDR}, which is the base address 
where the injector is allocated as an APB Slave component. This parameter must be updated accordingly on the \textbf{injector.h} file, so the calls are executed 
on the memory space the injector is allocated. In the case the \textbf{INJ\_BASE\_ADDR} is unknown, refer to section \fullref{} for further information.

The following section \fullref{software-drivers} presents the available calls by default on the drivers used to design an injection program.
However, they're limited in scope, thus, section \fullref{software-config} provides the information required to expand the drivers.

\subsection{Designing software for traffic generation and control}
\label{software-drivers}

The available function calls are organized into three categories; \fullref{software-drivers-program}, \fullref{software-drivers-setup} and \fullref{software-drivers-memfunc}.

\subsubsection{Injection program example}
\label{software-drivers-program}

The default driver files include an injection program example as the \textbf{inj\_program()} function, which can be reviewed on the \autoref{appendix-programexample}. 
Using this function, the main points can be understood to be able to design an injection program with the available drivers. 
This program is used for the program setup that generates transactions of a particular type, while inserting \textbf{DELAY} between transaction batches if desired. 
First, let's understand the variables the \textbf{inj\_program()} function admits as input.

\begin{itemize}
  \item \textbf{DESC\_N\_BATCH}: The number of descriptors the injection program will contain. This variable must be at least 2 if inserting \textbf{DELAY} descriptors is desired and less than the available internal memory space implemented on the injector core. Review \fullref{} for further information.
  \item \textbf{DESC\_TYPE}: The type of traffic to be generated. At the moment, it allows the traffic generation of \textbf{READ}, \textbf{WRITE}, \textbf{READ\_FIX} or \textbf{WRITE\_FIX} transactions.
  \item \textbf{INJ\_QUEUE}: Enables the \textbf{QUEUE\_MODE\_EN} parameter during the configuration of the injector, setting the injector in an execution loop of the injection program.
  \item \textbf{SIZE\_RD\_WR}: Number of bytes that are accessed during the execution of a descriptor iteration of \textbf{READ}, \textbf{WRITE}, \textbf{READ\_FIX} or \textbf{WRITE\_FIX} type.
  \item \textbf{DESC\_ATTACK\_ADDR}: Pointer of the starting address where the traffic being generated accesses.
  \item \textbf{SIZE\_DELAY}: Number of clock cycles to be on standby between the execution of the descriptors of type \textbf{DESC\_TYPE} descriptor types. If zero, no \textbf{DELAY} descriptors will be added to the program.
\end{itemize}
\vspace{15px}

At the start of the function body, there's a number of initialization variables that use specific structs of the driver. 
These are used to set up the appropriated configuration of the injector and manage the different descriptors to upload on the injector as an injection program. 
The \autoref{table:structs} compiles the different parameters to be set on each of these structs variables.

\begin{table}[h]
  \begin{tabular}{@{}p{0.15\linewidth}x{0.2\linewidth}p{0.45\linewidth}@{}}
    \toprule
    \multicolumn{1}{c}{Struct.parameter name} & \multicolumn{1}{c}{Admissible value range} & \multicolumn{1}{c}{Description of the parameter}                                                                             \\
    \cmidrule{1-3}
    desc\_ctrl.last           & {[}0-1{]}            & Last descriptor flag on injection program.                                                                                                                         \\
    desc\_ctrl.type           & {[}0-6{]}            & Descriptor type encoding. Use the descriptor name type listed on \autoref{table:descriptors} while adding \textbf{INJ\_OP\_} as a prefix.                          \\
    desc\_ctrl.irq\_compl\_en & {[}0-1{]}            & Send an interruption through the APB network on descriptor execution completion.                                                                                   \\
    desc\_ctrl.count          & {[}0-63{]}           & Set a number of descriptor execution iterations equal to \textbf{COUNT}+1.                                                                                         \\
    desc\_ctrl.size           & {[}1-524288{]}       & Set the number of bytes to access or the number of clock cycles to wait on standby depending on the descriptor type on a single execution iteration.               \\
    desc\_delay.ctrl          & Struct               & Nested struct parameter of type \textbf{desc\_ctrl}.                                                                                                               \\
    desc\_rd\_wr.ctrl         & Struct               & Nested struct parameter of type \textbf{desc\_ctrl}.                                                                                                               \\
    desc\_rd\_wr.act\_addr    & {[}0x0-0xFFFFFFFF{]} & Variable that sets the \textbf{ATTACK\_ADDR} on descriptors that use the descriptor type \textbf{READ}, \textbf{WRITE}, \textbf{READ\_FIX} or \textbf{WRITE\_FIX}. \\
    \bottomrule
  \end{tabular}
\caption{Struct types used on the injector configuration and programming drivers.}
\label{table:structs}
\end{table}

The \textbf{inj\_config} struct has been omitted from the \autoref{table:structs} due to containing the same parameters previously explained on 
\autoref{table:configuration}.

After setting the desired structs, namely an \textbf{desc\_rd\_wr} for the traffic generation, \textbf{inj\_config} for the injector configuration and 
\textbf{desc\_delay} in case the insertion of \textbf{DELAY} descriptors is desired between transaction descriptors. 
The function proceeds by loading the descriptors in execution order to the injector and, at last, the injector configuration using setup calls which are 
explained on \fullref{software-drivers-setup}.


\subsubsection{Injection setup functions}
\label{software-drivers-setup}

The setup functions are used to parse the struct variables to actual 32-bit words following the machine format the injector understands, to then be written 
on the injector using the memory access functions presented on \autoref{software-drivers-memfunc}, or also to perform control functions like a status check.
The following list compiles all available setup functions.

\begin{itemize}
  \item \textbf{inj\_setup()}: It accepts a \textbf{inj\_config} struct variable pointer as input, which reorganizes and writes on the injector. This function is commonly used after having uploaded the injection program, since this may start the execution of the program.
  \item \textbf{inj\_reset()}: Doesn't have any input, setting the configuration to zeros, except the \textbf{RESET\_SW} parameter that is set to 1, resetting the all injector's components including the loaded injection program.
  \item \textbf{inj\_check\_run()}: Doesn't have any input. This function returns 1 if the actual injector configuration has \textbf{ENABLE} parameter asserted, meaning it is running.
  \item \textbf{setup\_descriptor\_control()}: It accepts a \textbf{desc\_ctrl} struct variable pointer as input, which reorganizes and writes on the injector as word of a descriptor. This call should never be used on an injector program directly, since it is more organized to use it on a specific descriptor type setup calls.
  \item \textbf{setup\_descriptor\_delay()}: It accepts a \textbf{desc\_delay} struct variable pointer as input, which reorganizes and writes on the injector as word of a descriptor of \textbf{DELAY} type.
  \item \textbf{setup\_descriptor\_rd\_wr()}: It accepts a \textbf{desc\_rd\_wr} struct variable pointer as input, which reorganizes and writes on the injector as words of a descriptor of \textbf{READ}, \textbf{WRITE}, \textbf{READ\_FIX} or \textbf{WRITE\_FIX} types.
\end{itemize}
\vspace{15px}


\subsubsection{Injection memory access functions}
\label{software-drivers-memfunc}

The memory access functions are functions used to execute read and writes on the APB memory space allocated for the injector.
A difference between the memory access and the setup functions is that these only work with 32-bit data words without knowledge on the actual application of the 
access.
It is recommended, for best organization, to use the following memory access functions through setup functions and not directly on an injection program.

\begin{itemize}
  \item \textbf{inj\_read\_reg()}: Returns a 32-bit word with the data read from the APB \textbf{INJ\_BASE\_ADDR} + (\textbf{entry}*4) address.
  \item \textbf{inj\_write\_reg()}: Writes the input integer \textbf{value} on the APB \textbf{INJ\_BASE\_ADDR} + (\textbf{entry}*4) address.
\end{itemize}
\vspace{15px}

For the variable input \textbf{entry} of these functions, it can be used the \textbf{INJ\_POINTER\_CONFIG} and \textbf{INJ\_POINTER\_PROGRAM} constants to 
access the APB configuration or input a descriptor word to the internal program memory of the injector, respectively.

The memory access functions have a limitation set where the access is restricted to the APB memory space used to allocate the injector, providing low-level 
protection against erroneous address access while using the injector drivers. 
This is achieved by using the variables \textbf{APB\_MEM\_SPACE} and \textbf{INJ\_BASE\_ADDR} as a limiter and pointer to the base address of the injector 
allocation on the APB memory space.
Both variables are located on the driver file \textbf{injector.h}, like all the other constants of the drivers, and they must be updated accordingly to the implementation.



\newpage
\subsection{Bare-metal configuration}
\label{software-config}

In order to expand the existing drivers, it is required to acknowledge the APB registers of the injector's interface, so the injector can be configured, 
controlled and take actions on status, or also set the injection program to be executed.

\begin{register}{h}{APB configuration register}{\textbf{INJ\_BASE\_ADDR}+0x00}
  \label{reg:APB_control}
  \regfield{Reserved}             {25}{7}{{Unreachable}} % Reserved range (unnused)
  \regfield{freeze\_irq\_en}      {1}{6}{0}
  \regfield{irq\_err\_net\_en}    {1}{5}{0}
  \regfield{irq\_err\_core\_en}   {1}{4}{0}
  \regfield{irq\_prog\_compl\_en} {1}{3}{0}
  \regfield{queue\_mode\_en}      {1}{2}{0}
  \regfield{reset\_sw}            {1}{1}{0}
  \regfield{enable}               {1}{0}{0}
  \reglabel{Reset}
  \regfieldtext{25}               {25}{7}
  \regfieldtext{1}                {1}{6}
  \regfieldtext{1}                {1}{5}
  \regfieldtext{1}                {1}{4}
  \regfieldtext{1}                {1}{3}
  \regfieldtext{1}                {1}{2}
  \regfieldtext{1}                {1}{1}
  \regfieldtext{1}                {1}{0}
  \reglabel{}
  \regnewline
\end{register}

The functionality of each bit has been presented previously on \autoref{module desc-config}. 
Note that the \textbf{enable} bit may be set to 0 by the injector itself when a pipeline freeze occurs.

The access to these APB registers must be executed on data transactions of 32-bit, meaning there's no support for other data width accesses.
This is particularly important on the \fullref{reg:APB_desc_input}, since this register is where each word of the injection program 
must be written in order of descriptor execution and word, loads a 32-bit word per write access.

\begin{register}{h}{APB program memory feed register}{\textbf{INJ\_BASE\_ADDR}+0x3F}
  \label{reg:APB_desc_input}
  \regfield{}                     {32}{0}{{Exclusive write access}}
  \regfieldtext{32}               {32}{0}
  \regnewline
\end{register}

With these, all access to the injector core has been explained, but the programming itself that is explained on the following \fullref{software-program}.


\subsection{Bare-metal programming}
\label{software-program}
Each descriptor has its own different machine format for codifying the different parameters used during the execution. 
However, these have common parameters such as \textbf{LAST} and \textbf{IRQ\_EN}, which indicate the last descriptor of 
an injection program and set an APB interruption when the descriptor has been completed, respectively. 

Multiple word descriptors must be wrote into the \fullref{reg:APB_desc_input} in ascendant order, being first the word 0x00, then 0x04 and so on. 


\subsubsection{Descriptor DELAY}
\label{software-program-delay}
The \textbf{DELAY} descriptor sets a standby time equal to \textbf{SIZE} times \textbf{COUNT}+1. 
This descriptor only has one 32-bit word, depicted on Register \ref{reg:desc_delay}. 

\begin{register}{h}{DELAY descriptor word}{}
  \label{reg:desc_delay}
  \regfieldb{}              {19}{13}
  \regfieldb{}              {6}{7}
  \regfieldb{}              {1}{6}
  \regfieldb{}              {5}{1}
  \regfieldb{}              {1}{0}
  \regfieldtext{19}         {19}{13}
  \regfieldtext{6}          {6}{7}
  \regfieldtext{1}          {1}{6}
  \regfieldtext{5}          {5}{1}
  \regfieldtext{1}          {1}{0}
  \regfieldtext{SIZE}       {19}{13}
  \regfieldtext{COUNT}      {6}{7}
  \regfieldtext{IRQ\_EN}    {1}{6}
  \regfieldtext{00000}      {5}{1}
  \regfieldtext{LAST}       {1}{0}
  \regnewline
\end{register}


\subsubsection{Descriptors READ, WRITE, READ\_FIX and WRITE\_FIX}
\label{software-program-rd_wr}
The \textbf{READ} and \textbf{WRITE} descriptors, including its address fixed variants with the added suffix \textbf{\_FIX}, sets a total byte transfer of that 
type at least to \textbf{SIZE} times \textbf{COUNT}+1 bytes.
For each \textbf{COUNT}+1 execution of the descriptor, the \textbf{SIZE} is split in multiple batches on the request to the compatible network interface.
The maximum data batch transfer is set by a design variable called \textbf{MAX\_SIZE\_BURST}, which is specific to each implementation of the SafeTI. 
Review \fullref{} for further information about this and other generic variables.

The transfers are executed starting from the coded \textbf{ATTACK\_ADDR} address, even though it maintains the address fixed if the access type is a fixed 
address variant.
These descriptors have two 32-bit words, depicted on Registers \ref{reg:desc_rd_wr0} and \ref{reg:desc_rd_wr1}. 

\begin{register}{h}{READ, WRITE, READ\_FIX and WRITE\_FIX descriptor word}{0x00}
  \label{reg:desc_rd_wr0}
  \reglabelc{}
  \regfieldb{}              {19}{13}
  \regfieldb{}              {6}{7}
  \regfieldb{}              {1}{6}
  \regfieldb{}              {5}{1}
  \regfieldb{}              {1}{0}
  \reglabelc{}
  \regfieldtext{19}         {19}{13}
  \regfieldtext{6}          {6}{7}
  \regfieldtext{1}          {1}{6}
  \regfieldtext{5}          {5}{1}
  \regfieldtext{1}          {1}{0}
  \reglabelc{READ}
  \regfieldtext{SIZE}       {19}{13}
  \regfieldtext{COUNT}      {6}{7}
  \regfieldtext{IRQ\_EN}    {1}{6}
  \regfieldtext{00001}      {5}{1}
  \regfieldtext{LAST}       {1}{0}
  \reglabelc{WRITE}
  \regfieldtext{SIZE}       {19}{13}
  \regfieldtext{COUNT}      {6}{7}
  \regfieldtext{IRQ\_EN}    {1}{6}
  \regfieldtext{00010}      {5}{1}
  \regfieldtext{LAST}       {1}{0}
  \reglabelc{READ\_FIX}
  \regfieldtext{SIZE}       {19}{13}
  \regfieldtext{COUNT}      {6}{7}
  \regfieldtext{IRQ\_EN}    {1}{6}
  \regfieldtext{00101}      {5}{1}
  \regfieldtext{LAST}       {1}{0}
  \reglabelc{WRITE\_FIX}
  \regfieldtext{SIZE}       {19}{13}
  \regfieldtext{COUNT}      {6}{7}
  \regfieldtext{IRQ\_EN}    {1}{6}
  \regfieldtext{00110}      {5}{1}
  \regfieldtext{LAST}       {1}{0}
  \regnewline
\end{register}  
\begin{register}{h}{READ, WRITE, READ\_FIX and WRITE\_FIX descriptor word}{0x04}
  \label{reg:desc_rd_wr1}
  \regfieldb{}                {32}{0}
  \regfieldtext{32}           {32}{0}
  \regfieldtext{ATTACK\_ADDR} {32}{0}
  \regnewline
\end{register}






\newpage
\section{Appendixes}
\subsection{Example injection program}
\label{appendix-programexample}

\begin{lstlisting}[language=Verilog, caption=Example injection program function from the bare-metal driver file texttt{injector.c}.]
//*** Template Program Functions ***/

// Number of descriptors        / Type of transactions / Enable Queue mode                      /
// Transfer size per descriptor / R/W address          / Delay weigth after transfer descriptor /
void inj_program( unsigned int DESC_N_BATCH, unsigned int DESC_TYPE, unsigned int INJ_QUEUE, unsigned int SIZE_RD_WR, unsigned int DESC_ATTACK_ADDR, unsigned int SIZE_DELAY ) {
  unsigned int i  = 0;  // Loop variable
  inj_config inj_cfg;   // Injector configuration
  desc_ctrl  ctrl_word; // Descriptor control word
  desc_delay delay;     // Descriptor for DELAY type
  desc_rd_wr rd_wr;     // Descriptor for READ and WRTIE types

  // Check if the injector is running and reset it if it is.
  if(inj_check_run() != 0) {
    inj_reset();
    printf("WARNING: The injector was running before programming it. Thus, it has been reset.\n");
  }

  // Injector configuration
  inj_cfg.enable            = 1;
  inj_cfg.reset             = 0;
  if(INJ_QUEUE == 1)
    inj_cfg.queue_mode_en   = 1;
  else
    inj_cfg.queue_mode_en   = 0;
  inj_cfg.irq_prog_compl_en = 0;
  inj_cfg.irq_err_core_en   = 1;
  inj_cfg.irq_err_net_en    = 1;
  inj_cfg.freeze_irq_en     = 0;

  // Common descriptor setup
  ctrl_word.irq_compl_en    = 0;

  // Descriptor generation and setup
  for(i = 0; i < DESC_N_BATCH; i++) {

    ctrl_word.irq_compl_en = 0;     // Enable interrupt on descriptor completion
    ctrl_word.count   = 0;          // Number of repetitions of the descriptor

    if(i == DESC_N_BATCH - 1)       // Last descriptor bit
      ctrl_word.last  = 1;
    else
      ctrl_word.last  = 0;

    // Insert DELAY descriptor between two READ/WRITE descriptors if SIZE_DELAY has a value
    if(SIZE_DELAY != 0 & i%2 == 1) {
      ctrl_word.type  = INJ_OP_DELAY;
      ctrl_word.size  = SIZE_DELAY;
      delay.ctrl      = ctrl_word;
      setup_descriptor_delay(&delay);
    } else {
      ctrl_word.type  = DESC_TYPE;
      ctrl_word.size  = SIZE_RD_WR;
      rd_wr.ctrl      = ctrl_word;
      rd_wr.act_addr  = DESC_ATTACK_ADDR;
      setup_descriptor_rd_wr(&rd_wr);
    }

  }

  // Launch the injector with the configuration
  inj_run(&inj_cfg);
}

\end{lstlisting}



\end{document}
