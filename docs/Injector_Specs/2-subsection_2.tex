\newpage
\subsection{Bare-metal configuration}
\label{software-config}

In order to expand the existing drivers, it is required to acknowledge the APB registers of the injector's interface, so the injector can be configured, 
controlled and take actions on status, or also set the injection program to be executed.

\begin{register}{h}{APB configuration register}{\textbf{INJ\_BASE\_ADDR}+0x00}
  \label{reg:APB_control}
  \regfield{Reserved}             {25}{7}{{Unreachable}} % Reserved range (unnused)
  \regfield{freeze\_irq\_en}      {1}{6}{0}
  \regfield{irq\_err\_net\_en}    {1}{5}{0}
  \regfield{irq\_err\_core\_en}   {1}{4}{0}
  \regfield{irq\_prog\_compl\_en} {1}{3}{0}
  \regfield{queue\_mode\_en}      {1}{2}{0}
  \regfield{reset\_sw}            {1}{1}{0}
  \regfield{enable}               {1}{0}{0}
  \reglabel{Reset}
  \regfieldtext{25}               {25}{7}
  \regfieldtext{1}                {1}{6}
  \regfieldtext{1}                {1}{5}
  \regfieldtext{1}                {1}{4}
  \regfieldtext{1}                {1}{3}
  \regfieldtext{1}                {1}{2}
  \regfieldtext{1}                {1}{1}
  \regfieldtext{1}                {1}{0}
  \reglabel{}
  \regnewline
\end{register}

The functionality of each bit has been presented previously on \autoref{module desc-config}. 
Note that the \textbf{enable} bit may be set to 0 by the injector itself when a pipeline freeze occurs.

The access to these APB registers must be executed on data transactions of 32-bit, meaning there's no support for other data width accesses.
This is particularly important on the \fullref{reg:APB_desc_input}, since this register is where each word of the injection program 
must be written in order of descriptor execution and word, loads a 32-bit word per write access.

\begin{register}{h}{APB program memory feed register}{\textbf{INJ\_BASE\_ADDR}+0x3F}
  \label{reg:APB_desc_input}
  \regfield{}                     {32}{0}{{Exclusive write access}}
  \regfieldtext{32}               {32}{0}
  \regnewline
\end{register}

With these, all access to the injector core has been explained, but the programming itself that is explained on the following \fullref{software-program}.


\subsection{Bare-metal programming}
\label{software-program}
Each descriptor has its own different machine format for codifying the different parameters used during the execution. 
However, these have common parameters such as \textbf{LAST} and \textbf{IRQ\_EN}, which indicate the last descriptor of 
an injection program and set an APB interruption when the descriptor has been completed, respectively. 

Multiple word descriptors must be wrote into the \fullref{reg:APB_desc_input} in ascendant order, being first the word 0x00, then 0x04 and so on. 


\subsubsection{Descriptor DELAY}
\label{software-program-delay}
The \textbf{DELAY} descriptor sets a standby time equal to \textbf{SIZE} times \textbf{COUNT}+1. 
This descriptor only has one 32-bit word, depicted on Register \ref{reg:desc_delay}. 

\begin{register}{h}{DELAY descriptor word}{}
  \label{reg:desc_delay}
  \regfieldb{}              {19}{13}
  \regfieldb{}              {6}{7}
  \regfieldb{}              {1}{6}
  \regfieldb{}              {5}{1}
  \regfieldb{}              {1}{0}
  \regfieldtext{19}         {19}{13}
  \regfieldtext{6}          {6}{7}
  \regfieldtext{1}          {1}{6}
  \regfieldtext{5}          {5}{1}
  \regfieldtext{1}          {1}{0}
  \regfieldtext{SIZE}       {19}{13}
  \regfieldtext{COUNT}      {6}{7}
  \regfieldtext{IRQ\_EN}    {1}{6}
  \regfieldtext{00000}      {5}{1}
  \regfieldtext{LAST}       {1}{0}
  \regnewline
\end{register}


\subsubsection{Descriptors READ, WRITE, READ\_FIX and WRITE\_FIX}
\label{software-program-rd_wr}
The \textbf{READ} and \textbf{WRITE} descriptors, including its address fixed variants with the added suffix \textbf{\_FIX}, sets a total byte transfer of that 
type at least to \textbf{SIZE} times \textbf{COUNT}+1 bytes.
For each \textbf{COUNT}+1 execution of the descriptor, the \textbf{SIZE} is split in multiple batches on the request to the compatible network interface.
The maximum data batch transfer is set by a design variable called \textbf{MAX\_SIZE\_BURST}, which is specific to each implementation of the SafeTI. 
Review \fullref{} for further information about this and other generic variables.

The transfers are executed starting from the coded \textbf{ATTACK\_ADDR} address, even though it maintains the address fixed if the access type is a fixed 
address variant.
These descriptors have two 32-bit words, depicted on Registers \ref{reg:desc_rd_wr0} and \ref{reg:desc_rd_wr1}. 

\begin{register}{h}{READ, WRITE, READ\_FIX and WRITE\_FIX descriptor word}{0x00}
  \label{reg:desc_rd_wr0}
  \reglabelc{}
  \regfieldb{}              {19}{13}
  \regfieldb{}              {6}{7}
  \regfieldb{}              {1}{6}
  \regfieldb{}              {5}{1}
  \regfieldb{}              {1}{0}
  \reglabelc{}
  \regfieldtext{19}         {19}{13}
  \regfieldtext{6}          {6}{7}
  \regfieldtext{1}          {1}{6}
  \regfieldtext{5}          {5}{1}
  \regfieldtext{1}          {1}{0}
  \reglabelc{READ}
  \regfieldtext{SIZE}       {19}{13}
  \regfieldtext{COUNT}      {6}{7}
  \regfieldtext{IRQ\_EN}    {1}{6}
  \regfieldtext{00001}      {5}{1}
  \regfieldtext{LAST}       {1}{0}
  \reglabelc{WRITE}
  \regfieldtext{SIZE}       {19}{13}
  \regfieldtext{COUNT}      {6}{7}
  \regfieldtext{IRQ\_EN}    {1}{6}
  \regfieldtext{00010}      {5}{1}
  \regfieldtext{LAST}       {1}{0}
  \reglabelc{READ\_FIX}
  \regfieldtext{SIZE}       {19}{13}
  \regfieldtext{COUNT}      {6}{7}
  \regfieldtext{IRQ\_EN}    {1}{6}
  \regfieldtext{00101}      {5}{1}
  \regfieldtext{LAST}       {1}{0}
  \reglabelc{WRITE\_FIX}
  \regfieldtext{SIZE}       {19}{13}
  \regfieldtext{COUNT}      {6}{7}
  \regfieldtext{IRQ\_EN}    {1}{6}
  \regfieldtext{00110}      {5}{1}
  \regfieldtext{LAST}       {1}{0}
  \regnewline
\end{register}  
\begin{register}{h}{READ, WRITE, READ\_FIX and WRITE\_FIX descriptor word}{0x04}
  \label{reg:desc_rd_wr1}
  \regfieldb{}                {32}{0}
  \regfieldtext{32}           {32}{0}
  \regfieldtext{ATTACK\_ADDR} {32}{0}
  \regnewline
\end{register}


