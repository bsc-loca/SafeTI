\subsection{Descriptors specification}
\label{section:descriptors}
Descriptors are used to define, control, and monitor transactions in the Traffic Injector. Descriptor types supported by this module can be classified as \textbf{read}, \textbf{write} and \textbf{delay} descriptors. Furthermore, each transaction type has the possibility of starting a \textbf{burst transfer} by \textbf{not fixing} the Source and Destinations bits in the \textbf{Descriptor Control Word}.

\subsubsection{Descriptor format}

A single descriptor uses 20 Bytes of memory to be configured and monitored correctly.

\begin{table}[ht]
	\scriptsize
	\centering
	\begin{tabular}{ll}
		\hline
		Address Offset & Field
		\\
		\hline
		0x00 & Control Word
		\\
		0x04 & Next descriptor pointer
		\\
		0x08 & Destination base address
		\\
		0x0C & Source base address 
		\\
		0x10 & Status word
		\\
		\hline
	\end{tabular}
	\caption{Descriptor fields for configuration}
	\label{registers:descriptors}
\end{table}

On a general perspective, Figure \ref{figure:descriptor_system} shows the main block structure on the importance and flexibility that the descriptor system gives. Later on, we are going to describe each of the mentioned parameters in the diagram.\\\vspace{1cm}

\begin{figure}[H]
  \centering
  \includegraphics[width=16cm]{img/descriptor_system.pdf}
  \caption{Injector's descriptors structure. They are controlled from the APB Registers and written to the platform's main memory.}
  \label{figure:descriptor_system}
\end{figure}

\newpage
\paragraph{Control word} %subsubsubsection
{

The control word configures the main parameters of the descriptor.\\

\begin{register}{H}{Data descriptor control word}{ctrl~-~0x00}
    \label{desc_control}%
    \regfield{size}{19}{13}{0}%
    \regfield{count}{6}{7}{0}%
    \regfield{destfix}{1}{6}{0}%
    \regfield{srcfix}{1}{5}{0}%
    \regfield{irqe}{1}{4}{0}%
    \regfield{type}{3}{1}{0}%
    \regfield{en}{1}{0}{0}%
    \reglabel{Reset}\regnewline%
    \begin{regdesc}\begin{reglist}[Request~Depth]
        \item [size]Total size of data to be transferred from source to destination.
		Each bit defines a byte that is going to be sent/received. The minimum size is 4 bytes (A full address on a 32-bit address bus configuration).
        \item [count]Number of transaction repetitions.
        \item [dstfix]Flag: All data is to be written from the same (fixed) destination address.
        \item [srcfix]Flag: All data is to be read from the same (fixed) source address.
            to a second data line. When this bit is 0, a second line
            is available.
        \item [irqe]Enable interrupt on descriptor completion.
        \item [type]Descriptor type
		\begin{itemize}
  		\item 0: Read descriptor
  		\item 1: Write descriptor
  		\item 2: Delay descriptor
		\end{itemize}
        \item [en]Enable data descriptor 
		\begin{itemize}
  		\item 0: Disabled
  		\item 1: Enabled
		\end{itemize}
\end{reglist}\end{regdesc}\end{register}
}

\newpage
\paragraph{Next descriptor pointer} %subsubsubsection
It indicates the address for the next descriptor that has been set up. If the current descriptor is the last one in the queue, the \textbf{last} bit has to be '1'.\\

\begin{register}{H}{Next descriptor pointer}{next~-~0x04}
    \label{desc_next}%
    \regfield{addr}{31}{1}{0}%
    \regfield{last}{1}{0}{0}%
    \reglabel{Reset}\regnewline%
    \begin{regdesc}\begin{reglist}[Request~Depth]
        \item [addr]MSB of next descriptor start address.
        \item [last]Last descriptor in the descriptor queue.
		\begin{itemize}
  		\item 0: Not last descriptor.
  		\item 1: Last descriptor.
		\end{itemize}
 \end{reglist}\end{regdesc}\end{register}


\paragraph{Destination address} %subsubsubsection
It is used to indicate the destination address for the descriptor transaction. That means that only if we configure a \textbf{write} descriptor, this parameter will become relevant.\\

\begin{register}{H}{Destination base address}{dest~-~0x08}
    \label{desc_dst}%
    \regfield{addr}{32}{0}{0}%
    \reglabel{Reset}\regnewline%
    \begin{regdesc}\begin{reglist}[Request~Depth]
        \item [addr]Destination base address to which data is to be written.
 \end{reglist}\end{regdesc}\end{register}

\paragraph{Source address} %subsubsubsection

Is used to indicate the source address for the descriptor transaction. That means that only if we configure a \textbf{read} descriptor, this parameter will become relevant.\\
\vspace{0.3cm}
\begin{register}{H}{Source base address}{src~-~0x0C}
    \label{desc_src}%
    \regfield{addr}{32}{0}{0}%
    \reglabel{Reset}\regnewline%
    \begin{regdesc}\begin{reglist}[Request~Depth]
        \item [addr]Source base address to which data is to be written.
 \end{reglist}\end{regdesc}\end{register}

\newpage
\paragraph{Status word} %subsubsubsection
The descriptor status currently shows only if an error has occurred or if the transaction has been done. The injector status and the possible error flag gets propagated to the \textbf{Status APB Register.}\\

\begin{register}{H}{Descriptor status word}{sts~-~0x10}
    \label{desc_sts}%
    \regfield{reserved}{30}{2}{0}%
    \regfield{err}{1}{1}{0}%
    \regfield{done}{1}{0}{0}%
    \reglabel{Reset}\regnewline%
    \begin{regdesc}\begin{reglist}[Request~Depth]
        \item [err]Descriptor execution error status.
		\begin{itemize}
  		\item 0: No error.
  		\item 1: Error during execution.
		\end{itemize}
        \item [done]Descriptor completion without error.
		\begin{itemize}
  		\item 0: Not completed.
  		\item 1: Completed.
		\end{itemize}
 \end{reglist}\end{regdesc}\end{register}

